\documentclass[review,12pt,authoryear]{elsarticle}

\usepackage{amsmath}
\usepackage[utf8]{inputenc}
\usepackage{tipa, wasysym,graphicx,colortbl}
\usepackage{linguex,xspace,cgloss4e}
\usepackage{caption}
\usepackage{subcaption}
\usepackage[english]{babel}
\usepackage[dvipsnames]{xcolor}
\usepackage{comment}
\usepackage[section]{placeins}
%for margin notes
\usepackage{marginnote, setspace}
\newcommand{\evanote}[1]{%
   \marginnote{%
      \begin{spacing}{1}
         \vspace{-\baselineskip}%
         \color{OliveGreen}\footnotesize\singlespacing\itshape#1
      \end{spacing}
   }
}


\providecommand{\keywords}[1]{\textbf{\textit{Keywords}} #1}


%TIKZ
\usepackage{tikz}
\usetikzlibrary{decorations.pathreplacing,calc,shapes}
\usetikzlibrary{backgrounds,snakes,positioning}
\newcommand{\tikzmark}[2][-3pt]{\tikz[remember picture, overlay, baseline=-0.5ex]\node[#1](#2){};}
\tikzset{brace/.style={decorate, decoration={brace}},
 brace mirrored/.style={decorate, decoration={brace,mirror}},
}
\newcounter{brace}
\setcounter{brace}{0}
\newcommand{\drawbrace}[3][brace]{%
 \refstepcounter{brace}
 \tikz[remember picture, overlay]\draw[#1] (#2.center)--(#3.center)node[pos=0.5, name=brace-\thebrace]{};
}
\newcounter{arrow}
\setcounter{arrow}{0}
\newcommand{\drawcurvedarrow}[3][]{%
 \refstepcounter{arrow}
 \tikz[remember picture, overlay]\draw (#2.center)edge[#1]node[coordinate,pos=0.5, name=arrow-\thearrow]{}(#3.center);
}
% #1 options, #2 position, #3 text 
\newcommand{\annote}[3][]{%
 \tikz[remember picture, overlay]\node[#1] at (#2) {#3};
}
\usetikzlibrary{fadings}
\usepackage{hyperref}




\usepackage{changes} %add [final] to not show tracked changes
%\definechangesauthor[name={Eva}, color=OliveGreen]{eva}
\setremarkmarkup{(#2)}
\setdeletedmarkup{\color{red}{\sout{#1}}}
\usepackage[normalem]{ulem}
%\usepackage{draftwatermark}
%\SetWatermarkText{Authors' Draft}
%\SetWatermarkScale{3} 
\usepackage{afterpage} %helps figures to their own page
\bibliographystyle{model5-names}\biboptions{authoryear}
\newcommand*{\sectionformat}{\centering}

\date{\today}
\journal{Journal of Memory and Language}
\begin{document}

\begin{frontmatter}

\title{If You Want A Quick Kiss, Make It Count:\\
How Choice Of Syntactic Construction \\Affects Event Construal}

\author[ucsd]{Eva Wittenberg}
\ead{ewittenberg@ucsd.edu}
\author[ucsd,mit]{Roger Levy}

\address[ucsd]{Department of Linguistics\\University of California, San Diego\\9500 Gilman Drive\\La Jolla, CA 92093-0108}
\address[mit]{Department of Brain and Cognitive Sciences\\
Massachusetts Institute of Technology\\
77 Massachusetts Avenue\\
Cambridge, MA 02139-4307}



\begin{abstract}
When we hear an event description, our \added{mental} construal \deleted{of the event }is not only based on lexical items\deleted{ in the message}, but also on the message's syntactic structure. This has been well-studied in the domains of\deleted{ how listeners conceptualize} causation,\deleted{ roles of} event participants, or object conceptualization\deleted{properties, depending on syntactic choices}. Less studied are\deleted{ subtle aspects of} the construals of temporality and numerosity \deleted{of events }as a function of syntax. We present a theory of how \deleted{the} syntax \deleted{of an event description} affects the construal of event similarity and duration in a way that is systematically predictable from the interaction of mass\replaced{/}{ versus }count syntax and verb semantics, and test these predictions in six studies. \replaced{P}{We found that p}unctive events in count syntax{ (\emph{give a \replaced{kiss}{hug}})} and durative events in mass syntax{ (\emph{give advice})} are construed as taking less time than in transitive \deleted{verb }frame{ (\emph{\replaced{kiss}{hug}} and \emph{advise})}. Durative verbs in count syntax (\emph{give a talk})\replaced{, however, }{, on the other hand}\replaced{result in}{, yield a different pattern: Count syntax{ }, compared to simple transitive constructions{ (\emph{talk})}, leads to} a semantic shift\deleted{ in event structure}\replaced{, orthogonal to}{ and does not systematically affect} duration estimates. These results\replaced{ demonstrate}{ support our theory of} how syntactic and semantic structure together systematically affect event construal.

\end{abstract}


\begin{keyword}
Events \sep Lexical Aspect \sep Light Verb Constructions \sep Mass--Count distinction \sep Individuation \sep Linguistic Framing
\end{keyword}

\end{frontmatter}

\section*{\sectionformat Introduction}
When people talk to each other about what happened, they usually don't need to specify how long it took. Everybody knows from experience that a kiss lasts a few moments, a conference talk may carry on for about twenty minutes, and giving professional advice takes maybe half an hour, so there is typically no need to explicitly mention the duration. Duration is also usually not encoded grammatically. However, grammatical cues in event descriptions often significantly influence other aspects of event representations in listeners, such as causation, event structure, and the semantic roles of event participants \citep{fausey2010, wittenberg2014sorting,JohnsonGoldberg2013}. It would be all the more interesting, thus, if very subtle grammatical choices were to reliably affect how long listeners think an event takes. 

In this paper, we explore how encoding event descriptions in simple verbs \emph{(to kiss, to advise)} versus count or mass noun light verb constructions \emph{(to give a kiss, to give advice)} has repercussions on the temporal encoding of these events. Based on the fundamental observation that the reference properties of syntactic objects can change the reference properties of the whole predicate (\citeauthor{Krifka1992}, \citeyear{Krifka1992}), we predict that nominalizing an event can help dividing experience into countable units, influencing duration estimates in a way that is systematically predictable from the interaction of verb semantics and nominal syntax.

This hypothesis was inspired by a previous study on how events are individuated, depending on mass and count syntax. \citet{barner2008events} found that using count syntax \emph{(to do climb\underline{s})}, but not mass syntax \emph{(to do climbing)}, affects how events are quantified; and that \added{atomic, }punctive events (\emph{kissing, kicking}) are more readily quantified by counting over individual subevents (\emph{kisses, kicks}) than \added{non-atomic, }durative events. This is in line with the Number Asymmetry hypothesis \citep{Barner2006}: whereas count syntax specifies individuation, mass syntax is underspecified. 

If it is true that mass and count syntax contribute to event individuation, then we should expect predictable influences of mass or count syntax also on estimates of event duration. Figure \ref{aspectchange_model} provides a schema of our predictions. As explained above, we distinguish between two types of events: {\replaced{Atomic, telic}{bounded}}, mostly punctive events, like kissing or kicking; and {\replaced{non-atomic, atelic}{unbounded}}, mostly durative events, like talking or advising \citep[ see Footnote \ref{aspectualdistinction}]{Dowty1991,Vendler1957}. 

\afterpage{%
\begin{figure}[ht!]
\centering

%COORDINATES IN TIKZ (because I keep forgetting!): (1,2) = 1 to the right and 2 up from the origin of the picture
\usetikzlibrary{matrix}

\tikzfading[name=fade out,
	inner color=transparent!0,
	outer color=transparent!100]

\tikzfading[name=fade right,
	left color=transparent!0,
	right color=transparent!100]

\tikzfading[name=fade right two,
	left color=transparent!0,
	right color=transparent!70]

\begin{tikzpicture}
%\draw[style=help lines] (0,0) grid (10,10);
\draw[rounded corners,fill=red!50] (2.7,9.5) rectangle (6.3,9) node[pos=.5] {\bf{{ $\times$} \  \  { $\times$} \  \ { $\times$} \   \ { $\times$}}};
\path 	(0,9.5) 	node(iter) 		[rectangle, text width=2cm,draw=none] 		{\textbf{punctive events}}
		(5.2,9.8) 	node(expun) 		[rectangle, thick, text width=2cm,draw=none] 		{\emph{kiss}}
%		(3,9) 	node(long) 		[rectangle,rounded corners, draw,fill]		{{\tiny $\times$} {\tiny $\times$} {\tiny $\times$} {\tiny $\times$}}		
		(3,9.25) 	node(b) 											{}
		(6,9.25)	node(c)											{}
		(6,9.1)		node(ca)											{}
		%subevents
		(3,7) 	node(d) 		[rectangle,rounded corners, draw,fill=yellow!20]		{{$\times$}}
%		(3.5,7) 	node(e) 		[rectangle,rounded corners, draw,fill]		{{\tiny $\times$}}
		(4,7) 	node(f) 		[rectangle,rounded corners, draw,fill=yellow!40]		{{$\times$}}
%		(4.5,7) 	node(g) 		[rectangle,rounded corners, draw,fill]		{{\tiny $\times$}}
		(5,7) 	node(h) 		[rectangle,rounded corners, draw,fill=yellow!60]		{{$\times$}}
%		(5.5,7) 	node(i) 		[rectangle,rounded corners, draw,fill]		{{\tiny $\times$}}
		(6,7) 	node(j) 		[rectangle,rounded corners, draw,fill=red!50]		{{$\times$}}
		(6.5,6.3) 	node(expunlvc) 		[rectangle, text width=1.2cm,draw=none] 		{\emph{to give a kiss}}

		%text discrete substructure
		(8.5,9.5) 	node(discrete) 	[rectangle,text centered, text width=3cm,draw=none] 		{\textbf{atomic, telic}}
		%text lvc
		(6.7,8) 	node(clvc) 	[rectangle, text centered, text width=1.3cm,draw=none] 		{{{count syntax}}};

\draw[ultra thick,->] (ca) -- (j);
\draw[rounded corners, fill=blue!50,path fading = fade out] (2.6,5) rectangle (6.4,3.5);
\draw[rounded corners,fill=green!50] (4.9,2.1) rectangle (6.9,1.6);
\draw[rounded corners, fill=blue!50, path fading = fade right two] (2.2,2.1) rectangle (3.8,1.6);
\path 	(0,4.5) 	node(dur) 		[rectangle, text width=2cm,draw=none] 		{\textbf{durative events}}
		(4.5,4.8) 	node(exdur) 		[rectangle, text width=2cm,draw=none] 		{\emph{advise/talk}}
		(3,4.25) 	node(1) 											{}
		(6,4.25)	node(2)											{}
		(4.5,4.1)		node(lower)											{}
		%types of events
		(3,2) 	node(count) 		[rectangle,rounded corners, draw=none]		{\textbf{}}
		(3,1) 	node(exdurmass) 		[rectangle, text width=1.2cm,draw=none] 		{\emph{to give advice}}

		
		(6,2) 	node(mass) 		[rectangle,rounded corners, draw=none]		{\textbf{}}
		(6,1) 	node(exdurcount) 		[rectangle, thin, text width=1.2cm,draw=none] 		{\emph{to give a talk}}


		%text discrete substructure
		(8.5,4.5) 	node(cont) 	[rectangle,text centered, text width=3cm,draw=none] 		{\textbf{non-atomic, atelic}}
		%text lvc
		(6.3,3) 	node(clvc2) 	[rectangle, text centered, text width=1.3cm,draw=none] 		{{{count syntax}}}
		(2.9,3) 	node(mlvc) 	[rectangle,text centered, text width=1.3cm,draw=none] 		{{{mass syntax}}};
\draw[thick,snake=snake, path fading= fade right] (1) -- (2);
\draw[ultra thick,->] (lower) -- (mass);
\draw[ultra thick,->] (lower) -- (count);
\draw[thick,snake=zigzag] (5.1,1.8) -- (6.7,1.8);
\draw[thick,snake=snake] (2.5,1.8) -- (3.5,1.8);

    \matrix(dict)[matrix of nodes, below=4.5cm of exdur,
        nodes={align=center,anchor=center,text width=3.8cm},
        row 1/.style={anchor=south},
        column 1/.style={nodes={text width=2.3cm,align=right}}
    ]{\hline& \textbf{punctive count} & \textbf{durative mass} & \textbf{durative count} \\
        \textbf{Predictions:} & \emph{kiss $\rightarrow$ give a kiss} & \emph{advise $\rightarrow$ give advice} & \emph{talk $\rightarrow$ give a talk}\\\hline
        temporal construal  & shorter & shorter & no prediction\\\hline
        number of events & fewer & same & fewer \\\hline
        event \ \ \ \ \ \ \ \ \ similarity & similar & similar & different	\\\hline\hline\\
    };

\end{tikzpicture}


\caption{Predictions of how mass versus count syntax interacts with verb semantics, with regards to event duration construals, number of events, and event similarity; see pp. \pageref{sec_predictions}ff..}
\label{aspectchange_model}
\end{figure}
\clearpage
}

Punctive events are distinct from durative events not only in that they are \added{conceptually} short and bounded by a natural end point \added{(telic)}, but also in that sentences in which they appear are often conventionally understood to describe several instances of the same punctive event, that is, they are understood iteratively \citep{paczynski2014events,barner2008events,Kim2015}: For instance, you may find that \emph{John kissed Mary} evokes the image of not one, but multiple kisses, each of which can be categorized as a subevent of kissing. Thus, punctive events can have a distinct substructure. Durative events, in contrast, are \replaced{atelic}{unbounded} and \deleted{continuous over time}, they do not possess a distinct substructure, and they do not receive an iterative reading, even if the duration of the event is explicitly extended beyond a conventionally accepted time frame (cf. \emph{Senator Cruz talked all night}). 

Many of the aforementioned events, like \emph{kiss, advise} or \emph{talk}, can either be expressed as transitive verbs, or as so-called light verb constructions. In light verb constructions, the verb contributes little semantics beyond tense, number agreement, and aspect, while the meaning of the expression comes from the deverbal noun \citep{Jespersen1954,Wiese2006,Jackendoff1974,Brugman2001,Butt2003,butt2010}. These light verb constructions appear either with count syntax, such as \emph{to give a kiss} and \emph{to give a talk}, or mass syntax, such as \emph{to give advice}. Thus, light verb constructions offer us an opportunity to study the interaction of verb type and mass versus count syntax with an existing alternation, as opposed to unusual constructions such as \emph{to do climbs} \citep{barner2008events}\added{, or using novel lexical items \citep{WellwoodOUP}}: Light verb constructions, like \emph{to give a kiss}, and their full verb counterparts, like \emph{to kiss}, are in a relationship of syntactic alternation with minimal difference in meaning \citep{glatz2006funktionsverbgefuge,Allerton2002}. In our study of punctive and durative events, we use light verb constructions with \emph{give}, which is itself \replaced{telic}{bounded} \citep{newman1996}.

\subsection*{\sectionformat The mass--count distinction and verbal aspect}

\added{Ever since \cite{bach1986algebra}, linguistic theory has been fascinated by the parallels between kinds of objects vs. substances on the one hand, and kinds of atomic vs. non-atomic events on the other hand  \citep{Krifka1992,casati2008event,quine1969,rothstein2008structuring,verkuyl1972compositional,jackendoff1991parts,Hale1993,Harley05howdo}. One of the defining differences between objects and materials is that labels for objects denote atomic units, which cannot be partitioned arbitrarily: Only a whole apple, not a piece of an apple, can be described with the count noun \emph{an apple}. A piece of an apple, on the other hand, will need to be further described with a quantifier or specific expression, such as \emph{slice of an apple}, or \emph{apple core}. Objects can also be individuated and counted (\emph{three apples}). Materials, however, are non-atomic, and can be partitioned in an arbitrary fashion: a quart of applesauce can be divided into many portions, yet each individual portion still denotes \emph{applesauce} \citep[and many others]{rips2015divisions,bale2009interpretation}. Introducing individuability to mass nouns, however,  is easily accomplished when they are quantized \citep[\emph{a bottle of wine, a quart of applesauce}; see][]{Krifka1992,Wiese2005}}.

\added{Events have the property of atomicity or non-atomicity, too: Some events are atomic, and some events are non-atomic. For example, if Mary kissed John, then she stopped just for a moment, and then started kissing him again, the pre-interruption kiss would be a new event, even if the people and location are the same: an event of kissing is atomic in that it cannot be broken apart. (Note also that the character of the start and end points is constitutive of the event: if there is not contact between a set of lips and a surface, with a a clearly defined onset and a clearly defined, voluntary or involuntary, offset, the term \emph{kiss} does not apply.) This not true for all events (or processes, see \citeauthor{WellwoodOUP}, \citeyear{WellwoodOUP}): If the president talked to a crowd, stopped for a moment, and started talking again, it could still be the same event of talking. Similarly, {advising} can be partitioned and spread over many advising sessions, but the overarching event of advising is the same, as long as there is some degree of spatial or character continuity \citep{anderson1983accessibility,zacks2001event,magliano2011impact}. Talking and advising are thus non-atomic: they can thus be broken up and still count as the same talking and advising events.}

\added{The atomicity and non-atomicity of events is highly correlated with notions of telicity, boundedness, and aspect in verbs or predicates, as well as the distinction between events and processes in some approaches.}\footnote{\label{aspectualdistinction}Telic events are said to involve some kind of natural endpoint (\citeauthor{andersson1972aktionalitat}, \citeyear{andersson1972aktionalitat}; \citeauthor{garey1957verbal}, \citeyear{garey1957verbal}; \citeauthor{bauer1970english}, \citeyear{bauer1970english}; \citeauthor{klein1994time}, \citeyear{klein1994time}; \citeauthor{Vendler1957}, \citeyear{Vendler1957}, among others). This definition covers accomplishments, like \emph{to draw a circle}, but not all punctive events. The classical definition of punctive events is that they only take a moment in time. How long this moment takes, however, is underspecified: The duration of \emph{to sneeze} depends entirely on the sneezer; \emph{to explode} can conceptually take more than a few seconds; and the event described by \emph{to break} might last for a few minutes. Thus, temporal properties make up one part of the diagnostics; the other part is contributed by the intuition that there is a natural endpoint to a given event.

Linguistic diagnostics, such as test for aspectual types, are of limited help. Even the classic test for a durative, namely using a temporal \emph{for}-PP to detect {atelic} events, is not entirely reliable: \emph{John talked for an hour} is understood as one continuous event and thus classified as durative; \emph{John sneezed for an hour} is understood as iterated, and thus classified as punctive. But the time frame defined by the prepositional phrase matters immensely: \emph{John kissed Mary for a minute} is understood as continuous, and thus classified as durative; \emph{John kissed Mary for an hour} is likely understood as iterated, and would thus be classified as punctive (note, in contrast, the unavailability of an iterative reading for achievements: *\emph{John discovered the error for an hour}, e.g. \citeauthor{bott2010}, \citeyear{bott2010}). Further complicating the grammatical picture is that iterated punctive events pattern with (durative) activities in some tests (such as allowing for nonsubcategorized objects in reflexive resultatives or in \emph{out}-prefixation, or in some tense entailment relationships). When punctive events are understood non-iteratively, they pattern with (durative) achievements in some other tests (such as in onset repair readings; \citeauthor{kearns2000semantics}, \citeyear{kearns2000semantics}).

Thus, there are at least two world-knowledge factors at play in defining the aspectual class of a verb describing an event: the existence of a natural endpoint of an event and connected to that, the inherent duration of an event; and an event's tendency to occur several times in a row and so its availability for an iterative interpretation. In addition, linguistic diagnostics are not always straightforward and might involve pragmatic inferences that are beyond the lexical semantics of the verb.} \added{For the purpose of this paper, atomicity as described above will be the defining criterion for an event to be classified as either punctive (atomic, e.g., \emph{kiss}) or durative (non-atomic, e.g., \emph{talk, advise}), even though the overlap of terminologies is not perfect.}

\added{In this study, we focus on the interaction of count and mass syntax with punctive and durative events in a syntactic ditransitive frame provided by the verb \emph{give}, such as \emph{Mary gave Douglas a kiss}, \emph{the professor gave her student advice}, or \emph{the president gave a talk to the audience}.} Given the parallels between between mass/count syntax and verbal aspect, and given the \replaced{telic}{bounded} verb \emph{give}, we expect count and mass syntax to interact differently with punctive and durative event nouns: Count syntax with a punctive deverbal noun like \emph{giving a kiss} should pick out one single instance of \emph{kissing}; mass syntax with a durative deverbal noun like \emph{giving advice} should carve out a portion of \emph{advising}. If this is true, then the grammar conveys a subtle difference in event duration between the simple transitive verb and the light verb construction\added{, for any given event}: the light verb construction communicates an event that lasts a shorter time.

For durative events in count syntax in combination with the \replaced{telic}{bounded} light verb \emph{give}, however, the predictions are not as straightforward. Just consider what happens when one packages substances like \emph{beer, glass, string, stone} or \emph{iron} into count syntax: In some cases, the count noun denotes portions (of arbitrary size) of the substance (\emph{a string, a stone, a beer}), but in other cases, the count noun phrase happens to encode \added{objects} or units that are related \added{to the mass noun}, yet \added{in an} arbitrary \added{way}, such as in \emph{a glass} or \emph{an iron}. Crucially, the resulting denotation for mass nouns in count syntax is variable, and each case conveys aspects of meaning that cannot be predicted from the intrinsic structure of the underlying substance  \citep{gordon1985evaluating,srinivasan2015concepts}. 

So what should we expect in the case of durative verbs entering count syntax, like \emph{talk} in \emph{give a talk}? If the analogy between durative verbs and substance nouns really holds, one would predict a certain degree of conventionalization, similar to the \emph{some glass} $\rightarrow$ \emph{a glass} case: Core parts of a given durative event would remain the same, but the event type would shift in meaning from verb to count noun construction, possibly closing a lexical gap in doing so \citep{Allerton2002,miyagawa1989light,Grimshaw1988,glatz2006funktionsverbgefuge}. For example, there is a strong intuition that \emph{giving a talk}, albeit still retaining the core meaning of utterance production, is conceptually further from \emph{talking} than \emph{giving a kiss} is from \emph{kissing}.

But then, if it is true that durative events in count syntax undergo a conceptual shift in event kind, predictions about event duration are up in the air, since the change induced by count syntax would be orthogonal to changes in temporal conceptualization. In this paper we explore both sides: Whether event duration estimates are modulated by the introduction of mass and count syntax, and whether there is an effect on how similar events are judged as depending on the syntactic construction. In the next section, we will discuss the link between event representation and linguistic encoding in more detail.

\subsection*{\sectionformat Event construal via linguistic encoding}
How linguistic framing influences people's event conceptualization, memory, and recall has long been a topic of interest in science, such as in behavioral economics  \citep{kahneman1977intuitive, roy2005underestimating,kruger2004if,halkjelsvik2011read}, criminal justice \citep{loftus1987time,tse2004attention,macar1994controlled,boltz1995effects}, and cognitive science \citep{madden2003does,magliano2000verb}. These studies, however, were mainly concerned with behavioral or memory consequences of linguistic encoding accompanying visual scenes, and less with representational or grammatical issues.

In psycholinguistics, using grammatical alternations to study their influence on event construal started with \citet{gropen1991}, who found that subtle changes to event structure affected which form of the locative alternation people used in production (\emph{cover a surface with marbles} or \emph{dropping marbles onto a surface}). A later study confirmed the intuition that  syntactically omitting agents from an event description reduces how much blame is assigned to them \citep{fausey2010}. \replaced{D}{Finally, d}irectly related to the constructions used in this paper, we know that using a light verb construction with \emph{give} influences the construal of thematic roles \citep{wittenberg2014sorting,Wittenbergunderreview}. \added{And finally, there is evidence from a production study that the naturalness of event divisions predicts the choice of mass or count syntax \citep{WellwoodOUP}.}

The question of whether differences in event descriptions cause differences in duration estimates is less well studied. So far, there is only a small but growing experimental literature showing that verbal aspect or the verb itself influence duration estimates. For example, human locomotion events seen on video are remembered as taking longer when they are described as walking events than when they are described as running events \citep{burt1996effects}; and a Dutch study has shown that describing a short event in progressive verb aspect \emph{(is kissing)} makes people think that the kiss took longer than if simple present is used \citep{flecken2013grammatical}. \citet{pedersen2002differences}, in contrast, found only small effects of event descriptions on the duration estimates; however, their manipulation was on writing style and purposefully not as tightly controlled for semantic and syntactic factors as other studies. \added{Importantly, \citet[Study 4]{coll2011time} found that event duration estimates are tightly linked to aspectual nature of verbs.}

Thus, there is some evidence that linguistic choices influence the way people think about the temporal dimensions of events. Yet, it is not surprising that one should find this influence by grammatical means that by default operate in the temporal dimension, like aspect, or by choosing lexical items according to the speed of an action that they express. The alternation between transitive verbs and their light verb construction counterparts, however, affords a way to look beyond the more obvious aspects of how syntactic and semantic regularities work together to create a rich, full, and detailed representation of an event.

How could this interaction work? Most current theoretical approaches have accounted for the fact that some lexical items can appear both as nouns and verbs with the stipulation that prior to lexical insertion, their grammatical status is neutral. That means, a word like \emph{kiss} only acquires its grammatical category upon insertion into a syntactic tree; either way, regardless of whether it is inserted as a noun (such as \emph{He gave her a kiss}) or a verb (\emph{He kissed her}), it conceptually refers to the same kind of event \citep{halle1994some,barner2002no}. 

Taking this as a starting point, we should be able to observe a systematic interaction between syntax, the lexicon, and event construal. We assume that syntactic, conceptual, lexical, and phonological structure interact and predict upcoming features and structures every step of the way \citep{Jackendoff2007a,Jackendoff2002,GarrodPickering2003,levy2008,martin2016language}.\footnote{Another option would be a strictly linear model: A lexical item is inserted into a syntactic tree, its phonological form created, and its meaning read off by the comprehender, without any feedback loop to the semantic or conceptual system. In this case, both forms of the lexical items (e.g., \emph{kiss}, \emph{talk} or \emph{advise} as either noun or verb) are ontologically linked to the same event structure regardless of grammatical status, and it should not make a difference for duration estimates whether they appear as a verb or in either mass or count noun syntax, since the comprehender's lexicon is set up the same way -- \emph{kiss} conveys the same event regardless of whether it appears as verb or noun. Thus, this linear model would not predict any differences in duration estimates.} This architecture of the linguistic system allows for an interaction between representations on the levels of syntax (mass and count syntax), semantics (the \replaced{telicity}{boundedness} of \emph{give}), and event knowledge (how long kissing usually takes). Other approaches \citep{goldberg2003,Goldberg1995} explicitly allow for different meaning shades of grammatical constructions; presumably, the distinct meaning of a ditransitive construction as a \replaced{telic}{bounded} event would be even more straightforwardly predicted from this perspective. Crucially, both of these models would predict differences in duration estimates due to the interaction of verbal and nominal semantics and syntax.

\subsection*{\sectionformat Key theoretical predictions and the current studies}\label{sec_predictions}
The current studies investigate whether describing an event with mass or count syntax, as opposed to a simple transitive verb, affect\added{s} the construal of event duration, event similarity, and event repetitions in a comprehender. Figure \ref{aspectchange_model} shows a graphical representation of our theory.

Starting from the top of Figure \ref{aspectchange_model}, we look at punctive events like \emph{kissing}. Previous studies have shown that punctive events are often construed as occurring more than once \citep{paczynski2014events,barner2008events,Kim2015}, e.g., a comprehender might hear \emph{John kissed Mary} and imagine more than one kiss. In Figure \ref{aspectchange_model}, this is represented by four crosses, which are distinctive\added{, atomic} subevents within one \added{iterated, }bounded kissing event. Count syntax, according to our theory, should encourage event individuation in iterative events: In the case of our example, \emph{giving a kiss} describes only one \replaced{atom of a}{particular} kiss. This in turn should have repercussions in the construal of event temporality, leading to conceptualizations of shorter event duration. Another prediction is that punctive events in count syntax will be construed as consisting of fewer event iterations than in transitive syntax. 

The bottom of Figure \ref{aspectchange_model} visualizes our predictions for durative events. Durative events, represented by the continuous snake line, have no set endpoint (and in many cases, no set starting point either). They \replaced{are also non-atomic}{also do not have a continuous substructure}; for example, \emph{advising} is a process that consists of many points in time, but it is hard to conceptually delimit when advising starts and ends based on these time points. However, it is possible to carve out a particular portion of this process, and we claim that this is linguistically done when the event appears in mass syntax: \emph{to give advice} refers to a chunk of advising whose boundaries are limited (although not quite as strictly as in the case of \emph{to give a kiss}). In terms of event counts, the predictions are weaker than for the punctive event counts, since mass syntax does not aid in event individuation (although the \replaced{telicity}{boundedness} of the verb \emph{give} might).

When durative events occur in count syntax, such as in \emph{to give a talk} (bottom right on Figure \ref{aspectchange_model}), we hypothesize that, analogous to the cases of using mass nouns in count syntax (\emph{glass, iron} versus \emph{\underline{a} glass, \underline{an} iron}), there are arbitrary and unpredictable changes in the kinds of events these nouns then describe. For example, \emph{giving a talk} still retains a sense of utterance production, but in a very different context and with different event parameters (this difference is represented by a change in color, and the zigzag line, in Figure \ref{aspectchange_model}). This creates another prediction: \emph{Talking} and \emph{giving a talk} should be conceptually less closely related than punctive count and durative mass pairs. In terms of event counts, we predict that count syntax form will help event individuation and possibly lead to a reduction in event counts. Crucially, both event similarity and event count differences would operate entirely orthogonally from the construal of event duration.

We present six experiments that test these predictions. To test the claim that the construal of event duration is predictable from the interaction of mass versus count syntax, and verb semantics, we present two studies that elicited open estimates of event duration (Experiment 1a and 1b); \added{in order to ensure that the estimates obtained in Experiments 1a and 1b are not due to task difficulty}, we \added{then} present a temporal categorization experiment, in which participants categorized event descriptions into predefined time bins (Experiment 2). Then we present two studies that establish whether using count or mass syntax affects how many events people imagine: \added{Experiments 3a and 3b investigate whether the determiner "a" in ditransitive count syntax picks out one particular instance of an event, which would expound shorter temporal estimates in count syntax.} Finally we investigate whether describing durative events in count syntax indeed leads to conceptual shifts in event type, \added{which would explain the inconsistent pattern observed for durative events} (Experiment 4).

\subsection*{\sectionformat Stimuli used for all studies}
\label{sec_stims}
\begin{table}[h]
\hspace{-1cm}\noindent
\begin{tabular}{ r | l | l |  }
 			& 	\textbf{count} 				& 	\textbf{mass}  \\ \hline
\textbf{punctive} 		&	\emph{to kiss -- to give a kiss  }	& 		\cellcolor[gray]{.8}			\\ 
			&	\emph{to embrace -- to give an embrace} & \cellcolor[gray]{.8}\\
			&	\emph{to hug -- to give a hug} & \cellcolor[gray]{.8}\\
			& 	\emph{to kick -- to give a kick} & \cellcolor[gray]{.8}\\
			&	\emph{to poke -- to give a poke} & \cellcolor[gray]{.8}\\
			& 	\emph{to shake -- to give a shake} & \cellcolor[gray]{.8}\\
			&	\emph{to cuddle -- to give a cuddle }& \cellcolor[gray]{.8}\\ \hline
\textbf{durative}		&	\emph{to talk -- to give a talk}	&	\emph{to advise -- to give advice} \\
			&	\emph{to address -- to give an address}	&	\emph{to thank -- to give thanks}\\
			&	\emph{to lecture -- to give a lecture}	&	\emph{to assure -- to give assurance} \\
			&	\emph{to present -- to give a presentation}	&	\emph{to encourage -- to give encouragement} \\
			&	\emph{to speak -- to give a speech}	&	\emph{to recognize -- to give recognition} \\
			&	\emph{to scold -- to give a scolding}			&	\emph{to support -- to give support} \\
			&		 \cellcolor[gray]{.9}			&	 \emph{to assist -- to give assistance} \\\hline
			
\end{tabular}
\caption{Experimental item pairs used for all experiments.}
\label{ex_pairs}
\end{table}
\added{The verbs we used cluster together in a number of semantic and syntactic factors: For example, all events in the "punctive" category are \added{atomic,} short and in themselves \replaced{telic}{bounded}. They are, incidentally, also all contact verbs. In a light verb construction, they appear in count syntax. The events in the "durative count" category are all \added{non-atomic,} of medium duration, and \replaced{atelic}{unbounded}. Incidentally, they are also all actions of utterance towards others. When appearing in a light verb construction, they do so with count syntax. Their semantic common denominator may be best captured as social actions with the direct object as beneficiary. When appearing in a light verb construction, they take on mass syntax. Above and beyond other semantic aspects, however, \replaced{atomicity}{boundedness} of the verbs is the most important factor for the purpose of this paper, and we used the notions "punctive" and "durative" to distinguish between those categories of events.}

We used punctive verbs (\emph{kiss}) and durative verbs (\emph{advise, talk}) either in a transitive frame (\emph{After their first date, John kissed Mary}) or in a ditransitive light verb construction with the \replaced{telic}{bounded} verb \emph{give} (\emph{After their first date, John gave a kiss to Mary}) such as in Table \ref{ex_pairs} (see Appendix A for a full list of stimuli).\footnote{\added{In fact, \emph{talk} and \emph{speak} do not occur in transitive frames, but rather with prepositional objects \emph{(talk *(to) the students)}. In the context of this paper, however, we use "transitive" to mean two-place argument structures with either a Direct Object or a Prepositional Object.}} The ditransitive frame introduces a distinction between count syntax (\emph{give a kiss/talk}) and mass syntax (\emph{give advice}). We expected count syntax to force event individuation in punctive verbs, such that, when asked about event duration, people should judge the same event to be shorter in the ditransitive than in the transitive frame. For durative verbs, we predicted the same pattern for mass syntax, albeit to a lesser degree, because only the light verb \emph{give} would encourage \replaced{telicity}{boundedness}. For durative events that enter count syntax, we predicted a different pattern: Since there are no distinctive subevents that can be counted, applying count syntax to durative verbs should not lead to differences in duration. Instead, it should open the door to different event construals, orthogonal to changes in temporal structure (see Figure \ref{aspectchange_model} for an overview of predictions).

Of the twenty experimental items, seven encoded punctive events, e.g. \emph{kissing/giving a kiss}, six were durative events and could be used with count syntax, e.g. \emph{talking/giving a talk}, and seven encoded durative events that could be used with mass syntax, e.g. \emph{advising/giving advice}.\footnote{The missing item in the durative count category was \emph{condole -- giving condolences}, which we ultimately decided to exclude because \emph{condole} is used only very rarely as a verb in American English. In English, there are no punctive events in mass syntax with \emph{give}.} Note that in this last category, the noun describing the action (\emph{advice}) was not preceded by a determiner. None of the light verb constructions that we used can alternate between mass and count syntax (e.g. \emph{to give *(a) kiss; to give (*an) advice}). In short, there were three different verb alternations: punctive transitive vs. ditransitive with count syntax (punctive count); durative transitive vs. ditransitive count syntax (durative count); and durative transitive vs. ditransitive mass syntax (durative mass).

In all studies except for Experiment 4, the items were embedded in a sentence context. All sentences used the simple past. Each sentence included a temporal or local adjunct phrase, to encourage non-repeti\-tive readings (e.g. \emph{The professor advised her student on his paper \underline{yesterday}}). In addition to the experimental items, we also created 27 filler sentences.

\section*{\sectionformat Experiment 1a: Open estimates of event duration} \label{sec_expOE}
In Experiment 1a, we asked participants to rate how long events took. The events were described by simple transitive or light ditransitive constructions, and we were interested in whether the predictions displayed in Figure \ref{aspectchange_model} would be confirmed. If so, we expect punctive events like {kissing} to be estimated as taking less time in the ditransitive construction (\emph{giving a kiss}). We should also observe the same trend for durative events when they occur in mass syntax (\emph{advising -- giving advice}), but not durative events in count syntax (\emph{talking -- giving a talk}). Experiment 1b served as a replication, with event category as a between-subjects factor.

\subsection*{\sectionformat Methods}
\subsubsection*{Participants} \label{sec_partOE}
We recruited 100 unique individuals on Amazon Mechanical Turk, an online
crowd-sourcing tool. Mechanical Turk allows access to a large number of study participants, who participate anonymously for reasonable compensation, in our case for about \$6 an hour \citep{buhrmester2011amazon,crump2013evaluating}. Our participants had IP addresses within the United States and were self-reported native speakers of English.

\subsubsection*{Stimuli}\label{sec_stimOE}
We created two lists out of the sentences described above and distributed the experimental items across them with a fully within-subjects Latin-square design, such that each participant saw each sentence in only one of the two constructional forms (transitive, or ditransitive light verb). The fillers were the same across lists.

\subsubsection*{Procedure}\label{sec_procOE}
Participants read each sentence and then estimated how long the event described in the sentence probably took. The exact instructions can be found under \url{https://github.com/ewittenberg/QuickKissing}.

For each item, participants were able to enter their estimated event duration in a set of three text boxes, one for hours, one for minutes, and one for seconds, e.g., a participant could respond ``1 hour(s), 13 minute(s), 7 second(s)". Empty boxes were treated as a response of zero for that unit of time. Completing the study took about 17 minutes on average.

\subsection*{\sectionformat Results}\label{sec_resOE}
Responses in which the estimated duration was zero were discarded. This affected less than .1\% of the data. Since effects of grammatical structure on event duration would be likely to operate proportionally to intrinsic event duration, we transformed all responses to log-seconds for purposes of data summarization and analysis. Figure \ref{fig_resOE} shows the pattern of results, with responses back-transformed to hours, minutes, and seconds, for convenience of interpretation. \added{In all figures in the paper, bar plots show means and Standard Errors of by-subject means unless otherwise stated.}

\begin{figure}[ht!]
\centering
\includegraphics[width=\textwidth]{./Figures/Exp1a_AllEventsBar.pdf}
\caption{Experiment 1a: \replaced{Duration estimates per item pair.}{Punctive count and durative mass item pairs were rated to take less time when they were presented in ditransitive light verb frames. For durative count item pairs, the syntactic frame made no difference. }The y-axis is represented in log scale, but labeled with reader-friendly time estimates for convenience. \deleted{Error bars represent standard errors.}}
\label{fig_resOE}
\end{figure}

For punctive count events, using a ditransitive light verb construction instead of the transitive verb cut the time estimates in half, from about 40 to about 20 seconds. For durative count items, the effect was smaller (transitive $\mu$=31 minutes, ditransitive $\mu$=27 minutes), but in durative mass items, it was even stronger than for punctive count events (transitive $\mu$=50 minutes, ditransitive $\mu$=29 minutes).

For data from this and all following experiments (with the exception of Experiment 4, which has a different design), we conducted two types of statistical analysis. The first is a $2 \times 3$ ANOVA-style analysis of the main effect of construction (transitive or ditransitive light verb), the omnibus main effect of event category (punctive count, durative count, and durative mass), and the omnibus interaction between the two. The second is a set of planned pairwise tests of the effect of construction within each event category.  The reason for this latter set of planned tests is that the strength of evidence for an effect of construction for each event category is relevant to assessing the overall support of the data for our main hypothesis regarding effects of syntactic construction on event construal.  

In all analyses we used mixed-effects regression models (\citealp{baayen2008mixed}; in this experiment, linear mixed-effects models)  with \texttt{R}'s \texttt{lme4} package \citep{bates2014package}, using maximal random effects structure justified by the design (\citealp{barr2013}; where noted, random correlation parameters were dropped to ensure model convergence) and computing $p$-values through likelihood-ratio tests between models differing only in the presence or absence of the fixed-effect parameter(s) being tested. We used Helmert coding for both fixed-effects predictors, grouping punctive count and durative mass items together as one Helmert contrast pair, and their average contrasted with durative count items as the second Helmert contrast pair.\footnote{We implemented this coding in \texttt{R} as numeric predictors rather than as factors, which allows us to test main effects in the presence of the interaction through likelihood-ratio tests; see \citet{levy2014mainEffects}, among others, for details.}  

The top half of Table \ref{table_OEregr} shows the results of the $2\times 3$ ANOVA-style analyses (these analyses involved random by-participant intercepts and slopes for all fixed effects and random by-item intercepts and slopes for construction, with all random correlation parameters removed).  We see significant main effects of construction and event category; the interaction does not reach statistical significance.

The bottom half of Table \ref{table_OEregr} shows results of the planned pairwise tests within each event category (with random intercepts and slopes for both participants and items; no random correlation parameters needed to be omitted). Whenever punctive events appear in a count light verb construction with \emph{give} (ditransitive frame, such as\emph{ to give a kiss}), they are estimated to take \deleted{significantly} less time than when they appear in a simple transitive (\emph{to kiss}). \added{This effect was marginally significant.} \replaced{F}{The same pattern is apparent, but only marginally significant, f}or durative events in mass syntax (\emph{to give advice -- to advise})\added{, the same pattern was statistically significant}; the numeric pattern for durative events in count syntax is far from statistically significant (\emph{to give a talk -- to talk}).

\begin{table}[ht]
\centering
\begin{tabular}{lrrl}
  \hline
 & Df &  $\chi^2$ & \emph{p}-value \\ 
  \hline
construction          & 1 &  7.9 & 0.005 *** \\ 
  event category          & 2 &  23.02 & 0.000 ***\\ 
  construction $\times$ event category & 2 & 3.28 & 0.196 \emph{n.s.} \\    \hline
   punctive count -- construction    & 1 & 3.54 & 0.059 . \\    
   durative count -- construction    & 1  & 0.15 & 0.690 \emph{n.s.}\\ 
   durative mass -- construction   & 1 &  6.17 & 0.013 *\\ \hline
\end{tabular}
\caption{Table of likelihood estimation results for duration estimates in Experiment 1a, testing the main effects and their interaction (upper part), and the results of testing the main effect of construction in planned pairwise comparisons (lower part).}
\label{table_OEregr}
\end{table}



%PER ITEM FIGURE EXP 1
%\begin{figure}[ht!]
%\centering
%\includegraphics[height=1.1\textwidth,, angle=90]{../2_OpenEstimates/Results/SecondsBar.png}
%\caption{Experiment 1: Duration Estimates in log-seconds, per item pair. Error bars represent 95\% confidence interval.}
%\label{fig_resOE}
%\end{figure}

\subsection*{\sectionformat Replication: Experiment 1b}\label{repl_OE}

We replicated this study with 300 participants on Amazon Mechanical Turk by making event type (punctive count, durative count or durative mass) a between-subjects factor, reasoning that thinking about events on vastly different time scales might wash out sharper judgments. \added{For example, the literature on prospective time estimation shows that in events lasting more than a few moments, people count internally to "track time", while for very short events, this strategy is not used }\citep{grondin1999start,zakay1995attentional}\added{. Similarly, short events like kissing or hugging might be imagined by conjuring up an image of this event, while for longer events like talking or advising, abstract world knowledge might be employed to estimate duration. Experiment 1b served to exclude this potential mix of strategies in temporal estimation.} The exact instructions can be found under \url{https://github.com/ewittenberg/QuickKissing}.

\begin{figure}[ht!]
\centering
\includegraphics[width=\textwidth]{./Figures/Exp1b_AllEventsBar.pdf}
\caption{Experiment 1b (replication with event category as between-subjects factor): \replaced{Duration estimates per item pair. The y-axis is represented in log scale, but labeled with reader-friendly time estimates for convenience.}{Duration Estimates in log-seconds, per item pair. Error bars represent standard errors.}}
\label{fig_resOE1b}
\end{figure}

Figure \ref{fig_resOE1b} shows the pattern of results; as in Experiment 1a, punctive count item pairs were estimated to take roughly half as long when they were presented in ditransitive light verb frames (transitive  $\mu$=27 seconds, ditransitive $\mu$=14 seconds). Durative count items showed no trace of an effect of syntactic construction, with mean estimates of 27 minutes in the transitive frame and 31 minutes in the ditransitive frame. Durative mass items showed the same numeric pattern as punctive count items, with mean estimates of 76 minutes in the transitive frame and 54 minutes in the ditransitive frame, although this difference was not statistically significant.

Table \ref{table_OE_BetwSubj} shows the results of the regression analyses, conducted identically as those in Experiment 1a except that random by-subjects slopes for event category and its interaction with construction were excluded, since event category was a between-subjects manipulation. The top half of Table \ref{table_OE_BetwSubj} shows the results of the ANOVA-style analyses likelihood-ratio tests (since event category was both between subjects and between items, random by-subject slope for event category, or the random interaction, were not needed). Both main effects of construction and event category, and their interaction, were significant.

The planned pairwise comparison results are shown in the bottom half of Table \ref{table_OE_BetwSubj} (here, random correlation parameters did not need to be removed). As in Experiment 1a, whenever punctive events appear in a count light verb construction with \emph{give} (ditransitive frame, such as\emph{ to give a kiss}), they are estimated to take less time than when they appear in a simple transitive (\emph{to kiss}). The effect was not significant for durative events with mass syntax (\emph{to give advice -- to advise}), or for durative events with count syntax (\emph{to give a talk -- to talk}).

\begin{table}[ht]
\centering
\begin{tabular}{lrrl}
 \hline
 & Df &  $\chi^2$ & \emph{p}-value \\ 
  \hline
construction          & 1 &  4.98 & 0.026 * \\ 
  event category          & 2 &  27.52 & $<0.0001$ ***\\ 
  construction $\times$ event category & 2 & 6.10 & 0.047 * \\    \hline
   punctive count -- construction    & 1 & 9.64 & 0.026 ** \\    
   durative count -- construction    & 1  & 0.40 & 0.531 \emph{n.s.}\\ 
   durative mass -- construction   & 1 &  0.95 & 0.330 \emph{n.s.} \\ \hline
\end{tabular}
\caption{Table of likelihood estimation results for duration estimates in Experiment 1b, testing the main effects and their interaction (upper part), and the results of testing the main effect of construction in pairwise comparisons (lower part).}
\label{table_OE_BetwSubj}
\end{table}

\subsection*{\sectionformat Discussion of Experiments 1a and 1b}\label{sec_discOE}
As described above, the theory we proposed of how linguistic encoding and event type interact in event construal strongly predicted an effect of construction (transitive versus light-verb) on inferred event duration for punctive events whose light-verb encoding involves count syntax (\emph{kissing} vs. \emph{giving a kiss}): light-verb syntax should shorten inferred event durations.  This prediction was borne out in both Experiments 1a and 1b.  The theory predicted no systematic effect on inferred event duration for durative events encoded with count syntax, as this encoding requires more substantial, conventionalized \deleted{conventional} shifts  (\emph{talk} vs.\ \emph{give a talk}).  This prediction was also borne out: there was no trace of a systematic effect of construction on inferred event duration across Experiments 1a and 1b.  We suggested that the theory's predictions were less clear for durative events encoded with mass syntax (\emph{advise} vs.\ \emph{give advice}), and indeed our results across the two experiments were less clear. We found a significant shortening effect of light-verb syntax in Experiment 1a; in Experiment 1b, the same numeric pattern was evident but did not reach significance.

As a methodological note, one might notice that the duration estimates seemed fairly high. We know from previous studies that estimating the duration of an event is often influenced by its pleasantness or desirability \citep{roy2005underestimating,kahneman1977intuitive}. Although the numerical values of the estimated event durations are not crucial here, since we are only interested in the difference in estimates that is attributable to syntactic structure, it is conceivable that our participants' estimates were influenced by factors not due to the grammatical construction alone. For this reason, we decided to replicate this study as a categorization task with predefined time windows as answer options. The predefined time windows were based on previous answers, to validate the results obtained in Experiment 1a and 1b. 

\section*{\sectionformat Experiment 2: Categorizations of event duration}\label{sec_exCat}
This experiment gave participants the opportunity to estimate event durations without needing to come up with their own time estimates, and instead being able to select among predefined time bins for each individual event. 

\subsection*{\sectionformat Method}\label{sec_methodurative massat}
\subsubsection*{Participants} \label{sec_partCat}

We recruited 80 self-reported native speakers of English on Amazon Mechanical Turk with IP addresses within the United States.

\subsubsection*{Stimuli}\label{sec_stimCat}
We used the same sentences as in the previous studies. Participants read each sentence and then had to categorize the event for duration by clicking one of four options. These options were created from the quartiles of the empirical distribution of estimated durations of each item pair in Experiment 1b; thus, every item had different answer options, as in Examples (1) and (2):

\begin{enumerate}
\item[(1)] \emph{Laughing nastily, the thug kicked the victim. }\\How long did this take?
\begin{itemize}
\item [a)] up to 5 seconds 
\item [b)] between 5 seconds  and 13 seconds
\item [c)] between 13 seconds  and 10 minutes 
\item [d)] more than 10 minutes 
\end{itemize}
\item[(2)] \emph{The professor advised her student on his paper yesterday. }\\How long did this take?
\begin{itemize}
\item [a)] up to 25 minutes
\item [b)] between 25 minutes and 1 hour
\item [c)] between 1 hour and 2 weeks
\item [d)] more than 2 weeks
\end{itemize}
\end{enumerate}

The same lists as in Experiment 1a were used.
\subsection*{\sectionformat Results}\label{sec_resCat}

Figure \ref{fig_resCat} shows the proportional distribution of categorizations into the four time bins, and the proportion of counts in each category, depending on event category and grammatical construction. In general, we observed a tendency to categorize all events into the shorter time bins. More theoretically crucial, light verb constructions were categorized as being shortest.

To assess the strength of evidence for the light verb construction shifting responses systematically toward the shorter quartiles (and the strength of evidence for any interaction of such an effect with event category), we used a mixed-effects cumulative logit model using \texttt{R}'s \texttt{ordinal} package \citep{Christensen2015}.  For ordered response categories (in our case, four bins ranging from shortest to longest), a mixed-effects cumulative logit model specifies response probabilities for a given data point as a function of predictor variables. Instead of the intercept, ordered logit models provide a set of threshold parameters, which describe the boundaries from one bin to the next, and the probability of being drawn from one particular bin is estimated by the linear predictors with the inverse logit function (see Appendix B for an in-depth explanation of the mathematical underpinnings). The predictor variables in our model were construction (transitive or ditransitive), event category (punctive count, durative count, or durative mass), and their interaction. We Helmert-coded predictors and used maximal random effects structure as in Experiment 1a.  Under this coding, construction is coded with transitive=$-1$, ditransitive (light verb)=1.  Response categories were coded 1 to 4 in order of increasing duration.  For this model and predictor coding, if the light verb construction yields systematically shorter inferred event durations than the transitive construction, it should manifest as a significantly \emph{negative} parameter estimate for the fixed effect of construction.

Table \ref{tab_regrCat} shows the statistical results of the categorization task. The first three rows indicate the threshold coefficients from one bin into the next. The middle part of the table shows the regression coefficient for construction -- the --0.195 value indicating that the ditransitive construction is associated with shorter event durations -- and the results of the likelihood-ratio tests for the main effects and their interaction, using likelihood-ratio tests as in Experiments 1a and 1b.  Construction had a significant effect on duration estimates in the model comparison, and the interaction between construction and event category was marginally significant, but the main effect of event category was not. The lack of a main effect of event category is reassuring given that the time bins were constructed based on duration estimates from Experiment 1a. 

Finally, the bottom part of Table \ref{tab_regrCat} displays the results of analyses of the main effect of construction within each event category. For punctive count pairs, there was a significant difference in categorizations depending on construction. For durative count and durative mass items, this difference was not significant; however, a look at the $\beta$-estimates tells us that the effect of construction went in opposite directions for durative count items, compared to punctive count and durative mass items: whereas the ditransitive construction resulted in more ``shortest" categorizations for punctive count and durative mass items, this effect was absent (numerically: reversed) for the durative count items.

\begin{table}[ht]
\centering
\begin{tabular}{rrrrl}
  \hline
 & $\beta$ & SE &  &  \\ 
  \hline
shortest$|$short & -0.114 & 0.256 &  &  \\ 
  short$|$long & 2.201 & 0.265 &  &  \\ 
  long$|$longest & 4.669 & 0.320 &  &  \\ \hline \hline
  & Df &$\beta$ & LR.stat &  \emph{p}-value \\ \hline
 construction & 1 & -0.195 & 4.079 & 0.043 *\\ 
 event category & 2 & & 4.46 & 0.107 \emph{n.s.} \\ 
  construction $\times$ event category & 2 & & 5.21 & 0.070 . \\  \hline
  
   punctive count -- construction    & 1 & $-0.811$ & 4.995 & 0.025 * \\    
   durative count -- construction    & 1  & $0.290$ & 1.263 & 0.262 \emph{n.s.} \\ 
   durative mass -- construction   & 1 & $-0.607$ & 2.156 & 0.142 \emph{n.s.} \\ \hline

\end{tabular}
\caption{Experiment 2: Regression table for categorizations. }
\label{tab_regrCat}
\end{table}

\begin{figure}
\centering
\includegraphics[width=\textwidth]{./Figures/Exp2_CategorizationCategory.pdf}
\caption{Experiment 2: \replaced{Proportion of duration categories chosen, per event type and construction.}{For punctive count and durative mass items, participants chose the shortest durations in ditransitive light verb frame, and longer durations in the transitive verb frames. There was no significant difference for durative count items.}}
\label{fig_resCat}
\end{figure}


\subsection*{\sectionformat Discussion of Experiment 2}\label{sec_discCat}
This study gave participants the opportunity to estimate event durations using predefined time bins as choices, which might have been easier for them than coming up with time estimates on their own. The answer options in Experiment 2 were based on each item pair's averaged estimates from Experiment 1a, which, as we discussed above, seemed fairly high. Nevertheless, we found significant differences for punctive count events: When presented in ditransitive light verb constructions \emph{(give a kiss)}, they were estimated to last less time th\replaced{a}{e}n in transitive syntax \emph{(kiss)}. For durative count and durative mass events \emph{(talking/giving a talk, advising/giving advice)}, we did not observe such a difference, although the $\beta$-estimates indicate that durative mass events follow the same trend as punctive count events, with ditransitive structures pushing categorizations towards shorter bins. 

Interestingly, the largest proportion of choices in both grammatical constructions and all three event types fell to the \added{short or }shortest options. Given that the choice options were based on previously obtained quartiles, the answers should have been roughly equally distributed. Thus, our data contribute to the literature on over- and underestimation of event duration in an interesting way: Open guesses as in Experiment 1a and 1b tend to overestimate (at least for the event types we used here, which, unlike in classical studies on the planning fallacy, did not include unpleasant chore-like events), whereas predefined categorizations are closer to more realistic event durations.\footnote{We tested this explanation by giving participants shorter answer options that were not based on previous estimates. We found a much more equal distribution of answers. Since this study is tangential to the goal of this paper, it is not reported in detail, but the data are accessible under \url{https://github.com/ewittenberg/QuickKissing}.} 

As stated above, though, we were not \emph{per se} interested in how long events are estimated to take, but in the influence of mass and count syntax on estimated event durations. Crucially, as in Experiment 1a and 1b, we found that count syntax shortens the time estimate for punctive, but not durative events. We had predicted this pattern because punctive verbs are often interpreted iteratively, and by using count syntax, one picks out one particular subevent. For durative verbs with count syntax, however, there is no distinct subevent to be picked out: Thus, we hypothesized that durative verbs might undergo a larger conceptual shift \replaced{than}{when} when put in count syntax \emph{(to give a talk)} than do punctive count or durative mass verbs.

The next three studies test these \replaced{possible mechanisms of how temporal estimates are affected by syntax}{explanations}: Experiments 3a and 3b investigate \added{whether count syntax serves to single out one particular instance of an event. This could explain why events are imagined to be shorter in punctive count syntax. It would not, however, explain why durative count time-estimates were unaffected by changes in syntactic structure; in this case, we hypothesized that the lack of an effect in this condition is due to a conceptual shift between transitive verbs and count syntax, similar to the change from "iron" to "an iron". This conceptual shift would be orthogonal to event duration. We investigated the existence of such a conceptual shift in } \deleted{how many event repetitions are imagined by comprehenders, depending on the syntactic construction; and }Experiment 4\deleted{ asks whether using mass and count syntax introduced a conceptual shift between transitive and ditransitive mass versus count syntax}. 

\section*{\sectionformat Experiment 3a: Event repetition}\label{sec_expHowMany}

This section presents two studies that establish whether using count or mass syntax affects how many times an event is understood as occurring. 

The theoretically most crucial prediction applies to punctive events. Punctive events, like kissing, are often understood iteratively, even if their lexical semantics merely conveys one single, \replaced{telic}{bounded}, event \citep{Kim2015}. For example, people might interpret \emph{John kissed Mary} as him kissing her more than once. However, when presented in count syntax (\emph{to give a kiss}), only one kiss should be singled out. This would explain the consistently lower time estimates obtained in Experiments 1a, 1b, and 2.

Obviously, in order to be counted, a given event needs to be individuated, and we expect individuation to be easier in count than in mass syntax \citep{barner2008events}. So, for durative events in mass syntax, predictions are not quite as straightforward: \emph{The professor advised her student} consists of many individual advising situations. The \replaced{telic}{bounded} verb \emph{give} in \emph{The professor gave advice to her student} introduces a boundary to the advising event \citep{Krifka1992}; however, no definite article aids in the event individuation, so we might expect a weaker effect than for the punctive count events.

However, for durative events that can enter count syntax, we might predict the same trend as for punctive count events: \emph{talking} could convey more events than \emph{giving a talk}. Note that this prediction is independent of our second prediction, namely that there is a conceptual shift involved from a transitive durative verb to the same lexical item in ditransitive count syntax; and that any changes in event repetition counts would still be orthogonal to changes in event duration.

\subsection*{\sectionformat Methods}\label{sec_methodHowMany}
\subsubsection*{Participants} \label{sec_partHowMany}
For this study, we recruited 80 self-described English native speakers from Amazon Mechanical Turk.

\subsubsection*{Procedure}\label{sec_procHowMany}
Participants read each sentence and then noted how many events they imagined reading the sentence. The exact instructions can be found under \url{https://github.com/ewittenberg/QuickKissing}.

\subsubsection*{Stimuli}\label{sec_stimHowMany}
We used the stimuli described in Table \ref{ex_pairs}, counterbalanced across two lists. 

\subsection*{\sectionformat Results}\label{sec_resHowMany}
Our data spanned several orders of magnitude, (e.g., responses for punctive events in transitive frame ranged from 1 to 60), effects of construction are likely to operate proportional to intrinsic construal of numbers of events, as was the case with construed event durations in Experiments 1a and 1b.  Therefore we log-transformed event-count responses for purposes of data summarization and analysis. 

Mean count of events was lower for ditransitive light verb constructions in each event category. Figure \ref{fig_resCount}  shows the pattern of results, with log event counts back-transformed to raw event counts for convenience of interpretation: For punctive count events, using a ditransitive light verb construction instead of the transitive verb reduced the mean count from \replaced{1.7}{2.3} to \replaced{1.5 (log-averages)}{1.8}. These effects were also numerically present for durative count and durative mass items, but variances in these constructions were higher.

\begin{figure}
\centering
\includegraphics[width=\textwidth]{./Figures/Exp3a_EventCountsBar.pdf}
\caption{Count of imagined events in Experiment 3a. The y-axis is represented in log scale, but labeled with \replaced{back-}{non-}transformed counts \added{(log-averages)} for convenience. \deleted{Error bars represent standard errors.} }
\label{fig_resCount}
\end{figure}



\begin{table}[ht]
\centering
\begin{tabular}{lrrl}
  \hline
 & Df &  $\chi^2$ & \emph{p}-value \\ 
  \hline
construction          & 1 &  7.011 & 0.008 ** \\ 
  event category          & 2 &  5.326 & 0.069 .\\ 
  construction $\times$ event category & 2 & 0.016 & 0.992 \emph{n.s.} \\    \hline
   punctive count -- construction    & 1 & 7.187 & 0.007 ** \\    
   durative count -- construction    & 1  & 1.086 & 0.297 \emph{n.s.}\\ 
   durative mass -- construction   & 1 &  1.030 & 0.310 \emph{n.s.}\\ \hline
\end{tabular}
\caption{Table of likelihood estimation results for event counts in Experiment 3a, testing the main effects and their interaction (upper part), and pairwise comparisons (lower part).}
\label{tab_countsregr3}
\end{table}

Statistical analysis followed the same procedures as in Experiment 1a.  The upper part of Table \ref{tab_countsregr3} shows results of the $3 \times 2$ ANOVA-style analyses (here, all random correlation parameters were removed to ensure model convergence).  We find that construction had a significant main effect on event counts, event category was marginally significant, and their interaction was not.  The lower half of \added{Table} \ref{tab_countsregr3} shows results of the planned tests of the simple effect of construction within each event category (random correlation parameters did not need to be removed).  We see a significant effect of construction in punctive count pairs, but not in durative count or durative mass pairs.

\subsection*{\sectionformat Replication: Experiment 3b}\label{repl_HowMany}
We replicated this study by asking 80 native speakers on Amazon Mechanical Turk for the \added{number of} specific event types \added{imagined} after each sentence\deleted{s}, for example ``How many kisses did you just imagine?", ``How many talks did you just imagine?", or ``How many advices did you just imagine?" \added{(instead of, as in Experiment 3a, asking the generic ``How many \emph{events} did you just imagine?")}. Note that predictions for durative count items might be stronger in this study than in Experiment 3a, since we asked for the specific event \added{(e.g., talks), thus excluding any super- or subevents people might have imagined as well, such as "climbing the podium", "adjusting the microphone", etc}. We obtained a similar pattern of results (see Figure \ref{fig_resCount3b}).

\begin{figure}
\centering
\includegraphics[width=1\textwidth]{./Figures/Exp3b_EventCountsBar.pdf}
\caption{Count of imagined events in Experiment 3b. The y-axis is represented in log scale, but labeled with \replaced{back-}{non-}transformed counts \added{(log-averages)} for convenience \deleted{Error bars represent standard errors.}}
\label{fig_resCount3b}
\end{figure}

Data analysis procedures were identical to those in Experiment 3a. Table \ref{tab_countsregr3b} displays the results of the event count estimation test for the main effects, which were both significant, and their interaction, which was not. Pairwise comparisons (bottom part of Table \ref{tab_countsregr3b}) show main effects of construction for both punctive count and durative count, but not for durative mass events. 

\begin{table}[ht]
\centering
\begin{tabular}{lrrl}
  \hline
 & Df &  $\chi^2$ & \emph{p}-value \\ 
  \hline
construction          & 1 &  11.034 & 0.000 *** \\ 
  event category          & 2 &  8.297 & 0.016 *\\ 
  construction $\times$ event category & 2 & 1.001 & 0.606 \emph{n.s.} \\    \hline
   punctive count -- construction    & 1 & 5.391 & 0.020 * \\    
   durative count -- construction    & 1  & 4.063 & 0.043 *\\ 
   durative mass -- construction   & 1 &  2.234 & 0.135 \emph{n.s.}\\ \hline
\end{tabular}
\caption{Table of likelihood estimation results for event counts in Experiment 3b, testing the main effects and their interaction (upper part), and pairwise comparisons (lower part).}
\label{tab_countsregr3b}
\end{table}

\subsection*{\sectionformat Discussion of Experiment 3a and 3b}\label{sec_discHowMany}

This study investigated whether phrasing an event with mass or count syntax, instead of with a transitive verb, affects any iterative readings that are present. 

Crucially, we see a significant reduction in imagined number of events as a measure of implicit iterativity from transitive verb encoding to ditransitive light verb encoding -- but, as predicted, only consistently in punctive count events, and to a lesser degree, in durative count events. Th\replaced{u}{i}s, the more pronounced effect for durative count items in Experiment 3b, and the overall reduction in event counts in Experiment 3b, confirm our hypothesis that count syntax encourages individuating over subevents, leading to shorter event conceptualization. 

Contrary to what the previous literature has claimed, durative events resulted in iterative readings, as well; we attribute that to the fact that events like \emph{advising} or \emph{talking}, while not easily segmentable by pieces of advice or specific identifiable talking events, do carry an element of interactivity: There is a back-and-forth between the advisor or talker, and the addressee; and there might also be habitual interpretations available. Thus, it could be that our participants conceptualized the ``number of events'' question as ``number of subevents" -- which would explain the occasional ``more than one" answer for count syntax like \emph{give a talk}. 

However, we had hypothesized that count syntax might encourage a conceptual shift in durative verbs that acts orthogonally to any changes in temporal conceptualization. The last experiment investigates this possibility.

\section*{\sectionformat Experiment 4: Event Similarity}\label{sec_expES}
This study tested another prediction of our theory: that durative events in count syntax (\emph{to give a talk}) should be conceptually further apart from their transitive verb counterparts (\emph{to talk}) than punctive events in count syntax, or durative events in mass syntax. This prediction is drawn from the analogy to mass nouns, such as \emph{glass} or \emph{iron} into count syntax: In many cases, the count noun denotes objects that are conceptually related, yet different, from the noun in mass syntax, such as in \emph{a glass} or \emph{an iron}. If the durative events in count syntax behave in parallel, we should expect them to be conceptually further apart from their transitive counterparts than punctive events in count or durative events in mass syntax.

\subsection*{\sectionformat Methods}\label{sec_methodES}
\subsubsection*{Participants}
We recruited 40 self-described English native speakers from Amazon Mechanical Turk for this study.

\subsubsection*{Procedure}\label{sec_procES}
We asked participants to rate event similarity between transitive and ditransitive frames on a 7-point Likert scale where 1 indicated ``same event”, and  7, ``completely different event". The exact instructions can be found under \url{https://github.com/ewittenberg/QuickKissing}.

\subsubsection*{Stimuli}\label{sec_stimES}
We used the 20 item pairs shown in Table \ref{ex_pairs}, without the sentence context. In addition, we created 26 filler pairs that ranged from very close synonyms (\emph{repairing -- fixing}) to very different events (\emph{working -- being lazy}).

\subsection*{\sectionformat Results}\label{sec_resES}
For filler items, the average rating was 3.62 (SD=2.1), with the full range of the scale being used. 

Figure \ref{fig_resSimilarity} shows the rating results for critical pairs. Punctive events (\emph{to kiss} vs \emph{to give a kiss}) received a mean rating of 1.6 (SD=1.0). Pairs of durative verbs and durative verbs in mass syntax (\emph{to advise} vs \emph{to give advice}) received a similarly low rating (mean: 1.5, SD=.9). This means that for both punctive count and durative mass items, both constructions denote very similar events.  As predicted, difference ratings for pairs of durative verbs and durative verbs in count syntax (\emph{to talk} vs \emph{to give a talk}) were higher: 2.1 (SD=1.5).

\begin{table}[ht]
\centering
\begin{tabular}{lrrl}
  \hline
 & Df &  $\chi^2$ & \emph{p}-value \\ 
  \hline
  event category          & 2 &  6.819 &  0.033 * \\    \hline
   punctive count vs durative count    & 1 & 3.799 & 0.046 *   \\    
   durative count vs durative mass    & 1  &  4.869 & 0.027 * \\ 
   punctive count vs durative mass   & 1 &  0.000 &  0.984 \emph{n.s.}\\ \hline
\end{tabular}
\caption{Experiment 4: Table of likelihood estimation results for event similarity in Experiment 4, testing the main effect of event category (upper part), and pairwise comparisons between event categories (lower part).}
\label{tab_SimilarityModels}
\end{table}

Like in Experiments 1a, 1b, and 3, we used linear mixed models with maximal random effects structure (by-subjects intercepts and event-category slopes, by-items intercepts; no random correlation parameters needed to be removed). Table \ref{tab_SimilarityModels} displays the results from 3-level omnibus ANOVA-style analysis, as well as simple comparisons between all pairs of event categories. There was a significant main effect of event category in the omnibus analysis.  The pairwise comparisons show no significant difference between similarity ratings for punctive count and durative mass event pairs; durative count pairs, on the other hand, were rated significantly less similar to each other than pairs in the other two event categories.

\begin{figure}[t]
\centering
\includegraphics[width=\textwidth]{./Figures/Exp4_EventSimilarityBar.pdf}
\caption{Experiment 4: Punctive count and durative mass item pairs were rated more similar to each other than durative count item pairs. \deleted{Error bars represent standard errors.}}
\label{fig_resSimilarity}
\end{figure}

\subsection*{\sectionformat Discussion of Experiment 4}\label{sec_discES}
Experiment 4 confirms the intuition that while \emph{giving a kiss} and \emph{kissing}, as well as \emph{giving advice} and \emph{advising}, belong ontologically to the same event, \emph{giving a talk} and \emph{talking} are conceptually further apart -- albeit with considerable overlap. This result contributes to claims made in the literature that light verb constructions often help to close lexical gaps \citep{miyagawa1989light,Grimshaw1988,glatz2006funktionsverbgefuge}, and it draws our attention to interesting parallels to using count syntax on mass nouns  \citep{gordon1985evaluating,srinivasan2015concepts,Wiese2005}.

These results are crucial to our question of how using mass or count syntax affects the construal of event duration: If a durative lexeme in count syntax covers a conceptually different event than the same lexical item in a transitive verb frame, the conceptualization of event duration applies to the new concept, and thus any changes in duration estimates would be largely coincidental.

\section*{\sectionformat General Discussion}
We have presented a family of experiments that tested the hypothesis that describing an event with mass versus count syntax affects the construal of event similarity and duration in a way that is systematically predictable from the interaction of mass versus count syntax and verb semantics (see Figure \ref{aspectchange_model} on page \pageref{aspectchange_model}). These predictions were built on insights from formal semantics that has pointed out similarities between mass nouns and durative\added{, non-atomic} events on the one side, and count nouns and punctive\added{, atomic} events on the other side \citep{Krifka1992,jackendoff1991parts,Harley05howdo,WellwoodOUP,bach1986algebra,casati2008event}.

Specifically, we had predicted that punctive events in count syntax (\emph{give a \replaced{kiss}{hug}}) are construed as taking less time than in transitive verb frame (\emph{kiss}); this pattern was predicted because in punctive events, which are often interpreted iteratively \citep{Kim2015}, one \added{atomic }subevent is singled out by count syntax.

In durative events, the same pattern was predicted for events in mass syntax, such as \emph{giving advice}. In combination with the \replaced{telic}{bounded} verb \emph{give}, the mass noun carves out a limited portion of the event structure, leading to shorter event conceptualization. 

A different pattern was predicted for durative verbs in count syntax (\emph{give a talk} versus \emph{talk}): We expected durative events in count syntax to be semantically further from their verbal counterparts (\emph{to talk -- a talk}) than punctive events (\emph{to kiss -- a kiss}). \replaced{S}{s}ince these shifts were presumabl\replaced{y}{e} orthogonal to the temporal structure of the event, we did not make any predictions about duration estimates. These predictions stemmed from insights about the parallels between durative events and mass nouns: When some mass nouns are forced into count syntax (\emph{a glass; an iron}), there is a semantic shift from the mass noun meaning (\emph{glass; iron}).

Our experimental results broadly confirm these predictions. In Experiments 1a and 1b, we elicited open duration estimates, which were consistently lower for punctive events in count syntax and durative events in mass syntax than when they occurred in transitive frames; durative events in count syntax did not show this effect. Experiment 2 replicated these results by using the quartiles of each event's individual duration estimates obtained in Experiment 1a as answer choices; again, punctive count and durative mass ditransitive structures were judged to take less time than transitive structures, while there was no difference for durative events in count syntax. 

On our theory, a key factor for the reduction of duration estimates in punctive verbs was that people should imagine fewer events taking place than in the ditransitive count frame, due to the combination of a \replaced{telic}{bounded} verb (\emph{give}) and nominal count syntax (\emph{a kiss}). However, this effect should be weaker for durative events in either count or mass syntax when participants are asked for number of events. Experiment 3a confirmed these predictions. Experiment 3b showed that when people are asked for how many specific events they imagined -- how many talks, for example -- the reduction in event counts is significant even for durative events in count syntax. This confirms both the observation that event individuation is easier in count syntax \citep{barner2008events}, and that durative verbs behave similarly to mass nouns \citep{Krifka1992}.

Experiment 4 showed, consistent with our predictions, that durative events undergo a semantic shift in count syntax: The semantic differences between transitive and ditransitive frames were larger in durative count pairs (\emph{talk -- give a talk}) than in punctive-count pairs (\replaced{\emph{kiss -- give a kiss}}{\emph{hug --give a hug}}) or durative-mass pairs (\emph{advise -- give advice}). 

Thus, the shift from a transitive to a ditransitive frame has systematically predictable repercussions, depending on whether the event type was durative or punctive, and depending on whether the event was described with a mass or with a count noun. 

These results provide psycholinguistic evidence for the observation in formal semantics that reference properties of syntactic objects change the reference properties of the whole predicate (\cite{Krifka1992}; also see \cite{quine1969,verkuyl1972compositional}). In our case, the nominalization of eventive verbs using light verb constructions with a \replaced{telic}{bounded} verb helped to divide experience into countable units (for punctive verbs in count syntax and durative verbs in mass syntax), or introduced a semantic shift (for durative verbs in count syntax), similar to what happens with mass nouns in count syntax.

Our results also complement studies that suggest that people conceptualize events differently depending on subtle choices among syntactic alternations \citep{fausey2010,JohnsonGoldberg2013,wittenberg2014sorting} and that conversely, subtle changes in event structure result in changing preferences between syntactic alternations \citep{gropen1991}. This growing family of experiments, together with a  number of corpus studies \citep{bresnan2007,benor2006chicken}, contributes to our understanding how syntactic and semantic structures, as well as processing pressures\added{, such as a preference for using frequent, accessible lexical items}, lead to a speaker's decision between \replaced{two}{to} seemingly equivalent constructions.

These findings may also help interpret results from less explicit tasks. We know from behavioral, ERP, and MEG studies that light verb constructions are processed differently from non-light constructions \citep{wittenberg2014b,wittenberg2011,Briem2009,Pinangoinpress}. So far, however, only semantic role mismatches had been identified as a factor contributing to the processing difference \citep{wittenberg2014sorting,Wittenbergunderreview}. Based on the present work, we might hypothesize that the calculation of temporal structure also plays a role in the real-time processing of light verb constructions.

%Importantly, we do not claim that these temporal structures are always calculated explicitly. But the evidence suggests  that people consistently implicitly draw inferences about temporal duration based on grammatical structure, and they can rely on these inferences when questioned, like in the present studies.

%One potential, related, issue could have been the reliability of temporal estimates. Other studies, especially in the domain of productivity estimates, have found consistent temporal underestimations \citep{kahneman1977intuitive, roy2005underestimating,kruger2004if,halkjelsvik2011read}, so it is conceivable that people are simply bad at estimating how long events take. However, there are three arguments against the objection that our participants might have simply been bad at time estimates. First, we replicated our pattern of results consistently across studies. Second, we did not ask participants to estimate how long it would take them to do a task, but how long events generally take -- that is, there was no social pressure involved. And third, the pattern of results relevant to our question was not about about the general human (un)ability to make correct temporal predictions, but how grammatical frame and event structure interact; and the results both confirmed established analyses from formal semantics \citep{Krifka1992} and corroborated more recent psycholinguistic theories about the interaction of event structure and mass versus count syntax \citep{barner2008events}.

A question this paper raises is how well these results might generalize to other constructions\replaced{, for example,} {such as} \deleted{, or to idioms} other light verb constructions with \replaced{atelic}{unbounded} verbs. \deleted{In the literature on idiom comprehension there is a consensus that the lexical items in an idiomatic phrase such as \emph{to kick the bucket} are accessed individually as well as holistically. So if an idiom consists of a semelfactive verb (\emph{kick}) and a count noun (\emph{bucket}), and its semantically transparent counterpart is a semantic achievement (\emph{to die}) we would predict that \emph{John kicked the bucket} will be estimated to take a shorter time than \emph{John died}. Similarly, w}\added{W}e would predict that other light verb constructions with boun\-ded verbs would lead to the same effect; for example, \emph{John took a shower} should be estimated as taking less time than \emph{John showered}, whereas \emph{John had a shower }should affect estimated event durations to a lesser degree, since both verbs (\emph{to shower} and \emph{to have}) are \replaced{atelic}{unbounded}. \added{Another area where the interaction of count/mass syntax and verbal aspect might have repercussions for the temporal conceptualization of events are idioms. In the literature on idiom comprehension there is a consensus that the lexical items in an idiomatic phrase such as \emph{to kick the bucket} are accessed individually as well as holistically \citep{Sprenger2006,titone2014time,holsinger2013representing,cutting1997s,tabossi2009idiom}. So if an idiom consists of a semelfactive verb (\emph{kick}) and a count noun (\emph{bucket}), and its semantically transparent counterpart is a semantic achievement (\emph{to die}) we would predict that \emph{John kicked the bucket} will be estimated to take a shorter time than \emph{John died}.} These predictions remain to be addressed in future studies.

%However, we decided against including a set of such comparisons in the present paper. In idiom-verb pairs such as \emph{kick the bucket }vs. \emph{die}, the lexical items are completely different between idioms and their transparent counterparts. This introduces potential confounds into the comparison. A similar problem would have been introduced in a comparison across different light verbs. In the present studies, these potential confounds were absent: The verb was always the same (\emph{give} in the present studies), and the predicate only changed grammatical category (verb-noun alternation).

In sum, this paper showed that using mass versus count syntax affects the construal of event similarity and duration in a way that is systematically predictable from the interaction of mass versus count syntax and verb semantics. Our results confirm \replaced{observations}{hypotheses} from formal semantics \replaced{that the properties of objects and events are mirrored in count/mass syntax and verbal aspect}{and psycholinguistics, and connect to the literature on duration estimates in neighboring fields, such as behavioral economics. Thus, these studies}, and are advancing our understanding of the effects of syntactic choices on subtle aspects of event construal.

\newpage
\clearpage
\section*{Acknowledgments}
We would like to thank David Barner, Victor Ferreira, Ray Jackendoff, Titus von der Malsburg, Emily Morgan, Jesse Snedeker, David Townsend, Alexis Wellwood, and Heike Wiese for helpful discussion, and Suhas Arehalli for assistance in running the studies. This work was supported by NSF Grant IIS-0953870, NIH Grant HD065829, and an Alfred P. Sloan Fellowship to Roger Levy, as well as by an German Academic Exchange Service (DAAD) postdoctoral scholarship to Eva Wittenberg.
\clearpage
\section*{\sectionformat References}
\bibliography{QuickKissing,additional-refs}
\clearpage
\section*{Appendix A: Stimuli used for Experiments 1-3}
\renewcommand{\baselinestretch}{1.5}
Stimuli are shown in transitive frame. In the ditransitive light verb construction, the underlined verb would serve as the light noun, i.e., \emph{After their first date, Douglas kissed Mary} $\rightarrow$ \emph{After their first date, Douglas gave a kiss to Mary}.
\emph{
\begin{enumerate}
\item \textup{Punctive Count }
\begin{enumerate}
\item After their first date, Douglas \underline{kissed} Mary.
\item Laughing nastily, the thug \underline{kicked} the victim.
\item When they met up, Nathasha \underline{hugged} Cynthia.
\item Noam \underline{embraced} Jennifer before they split up.
\item Julius \underline{cuddled} his brother at bedtime.
\item Laura \underline{poked} Owen because he was so annoying.
\item Martin \underline{shook} the cocktail before he served it.
\end{enumerate}
\item \textup{Durative Count}
\begin{enumerate}
\item The professor \underline{lectured} on injustice yesterday.
\item The CEO \underline{talked} about his latest sales strategy last night.
\item The President \underline{spoke} about affordable education on Wednesday.
\item During the community art show, the composer \underline{presented} his latest work.
\item After the graduation ceremony, Tom Hanks \underline{addressed} the students.
\item The mother \underline{scolded} the child today.
\end{enumerate}
\item \textup{Durative Mass}
\begin{enumerate}
\item The professor \underline{advised} her student on his paper yesterday.
\item Yesterday, the nurse \underline{assisted} Dr. Kohler in the emergency room.
\item After the job interview, Sheila \underline{assured} Keith.
\item Before Kelly left for a year abroad, her friends \underline{encouraged} her.
\item After the devastating flood, the mayor \underline{recognized} the rescue workers.
\item His new girlfriend \underline{supported} Sam after his divorce.
\item At the summer party, the university official \underline{thanked} the academic community for their efforts.
\end{enumerate}
\end{enumerate}
}


\clearpage
\section*{Appendix B: The mixed-effects cumulative logit model used for Experiment 2}
Data were analyzed using R \citep{RCT2014}, specifically with mixed-effects cumulative logit model using the \texttt{ordinal} package \citep{Christensen2015}.  For ordered response categories $1, 2, \dots, N$, a mixed-effects cumulative logit model specifies response probabilities for a given datum $i$ as a function of predictor variables as follows.  There are $N-1$ linear predictors, each of the following form:
%
\begin{align*}
  \eta_{ij} & = \alpha_j + \sum_{k} \beta_k x_{ik} + \sum_{k} b_k z_{ik} & j \in \{1, 2, \dots, N-1\}
\end{align*}
%
where the $\{x_{ik}\}$ and $\{z_{ik}\}$ are fixed- and random-effects predictors respectively,  the $\{\beta_k\}$ and $\{b_k\}$ are fixed- and random-effects regression parameters respectively, and $\alpha_i$ is a \textsc{threshold parameter} for the boundary between the $j$-th and $(j+1)$-th response categories.  (The threshold parameters play a role analogous to the intercept in an ordinary mixed logit model.)  The linear predictors are related to cumulative probabilities $\{\gamma_{ij}\}$ through the inverse logit function:
%
\begin{align*}
  \gamma_{ij} = P(Y_i \leq j) = \frac{e^{\eta_{ij}}}{1+e^{\eta_{ij}}}
\end{align*}
%
The probability of datum $i$ having response category $j$ is thus
%
\begin{align*}
  P(Y_i = j) =
  \begin{cases}
    \gamma_{i1} & j=1\\
    \gamma_{ij} - \gamma_{ij-1} & j \in \{2, \dots, N-1\}\\
1-\gamma_{ij-1} & j=N
  \end{cases}
\end{align*}
%
That is, the $\{\gamma_{ij}\}$ carve up the interval $[0,1]$ into a set of category response probabilities, as illustrated in Figure~\ref{fig:cumulative-logit-models}.  In mixed-effects cumulative logit models, the random-effects regression parameters are assumed to be drawn from some multivariate normal distribution, the covariance matrix of which is estimated jointly along with  the threshold parameters and fixed-effects regression parameters (here, via Laplace-approximated maximum likelihood).

\begin{figure}[ht!]
  \centering
  \includegraphics[width=.8\textwidth]{Figures/cumulative-logit-models}
  \caption{Cumulative logit models.  The $N-1$ linear predictors $\{\eta_j\}$ induce a set of $N$ multinomial response category probabilities through the inverse logit transform. }
  \label{fig:cumulative-logit-models}
\end{figure}

\end{document}
